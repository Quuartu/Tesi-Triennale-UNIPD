\chapter{Conclusioni}

Il lavoro svolto per questa tesi ha avuto come obiettivo principale lo sviluppo di una soluzione SCADA integrata per la gestione di sistemi industriali, con focus sulle tecnologie utilizzate in REA Robotics. Nello specifico, la finalità è stata di implementare una piattaforma in grado di gestire in modo efficiente e sicuro il flusso di dati tra HMI e PLC, non solo permettendo di rispondere con successo alla task iniziale fornita dall'azienda, ma anche garantendo funzionalità operativa, sicurezza ed affidabilità del sistema. Il progetto ha visto come cardini l'utilizzo di strumenti e software come FT Optix™, C\#, SSMS e SQLite, per permettere la realizzazione di un'interfaccia che semplifica il monitoraggio e controllo dei macchinari. Nonostante sia stata raggiunta la task richiesta, sono emerse alcune criticità che meritano attenzione. Una delle principali sfide è stata la gestione separata dei due database, che ha comportato maggiore complessità nella progettazione da un lato e nella manutenzione futura dall'altro. Sarebbe auspicabile in futuro unificare la parte di database in un'unica implementazione generica che permetta una gestione più scorrevole, al fine di semplificare la mole di dati e codice presente e ridurre il rischio di incoerenze. Un'altra criticità riscontrata riguarda l'utilizzo dei recipe schema. La rigida dipendenza dalle regole imposte dal sistema di gestione di questi ultimi ha limitato la flessibilità del progetto, rendendo difficile adattare il sistema a future modifiche o implementazioni per altri impianti. In tal senso, una possibile evoluzione futura potrebbe prevedere il refactoring di alcuni script di codice per permettere la gestione dei dati più modulare e indipendente dalle regole imposte dai recipe schema, garantendo maggiore personalizzazione per i clienti. Guardando al futuro, sono diverse le opportunità di miglioramento e sviluppo. Per esempio, l'uso di C\# per ottimizzare la lettura delle variabili dei PLC consentirebbe un aggiornamento in tempo reale dei dati, una maggiore reattività del sistema alle modifiche e soprattutto meno tempo da dedicare lato programmazione, grazie a una gestione intelligente dei dynamic link attraverso una lettura diretta delle variabili. Inoltre, l'integrazione di Intelligenza Artificiale per l'analisi dei dati e per la progettazione dei pannelli potrebbe aprire nuovi scenari, consentendo una gestione ancora più efficiente dal lato software.

In conclusione, il progetto ha raggiunto gli obiettivi prestabiliti, complice un nuovo approccio nel contesto industriale di REA Robotics. Tuttavia, le criticità emerse durante lo sviluppo offrono sia spunti di riflessione, sia indicazioni per futuri miglioramenti. L'implementazione di soluzioni più moderne e flessibili, una gestione più modulare dei dati e l'introduzione di nuove tecnologie rappresentano sfide importanti per il futuro, con l'obiettivo di rendere questa tipologia di sistemi SCADA ancora più performante, sicura e adattabile alle esigenze in continua evoluzione dell'industria 4.0.
